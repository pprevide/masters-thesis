% preamble.tex, to be used with thesis.tex
% This contains the TeX definitions for layout, style, etc., as well as the first few pages of your thesis: title page, copyright page, approval page, abstract, acknowledgments, tables of contents, tables, and figures.
% The layout commands should give the correct margins according to the graduate division's guidelines

%%%%% TeX class and packages

\documentclass[12pt,oneside]{sfsuthesis}

\usepackage{amsmath,amsthm,amssymb,amsfonts,latexsym,graphicx,enumerate,setspace,verbatim,tocloft,rotating,url,caption, subcaption, appendix}
%\usepackage{color}                    % For creating colored text and background
%\usepackage{hyperref}                 % For creating hyperlinks in cross references
% other possibly useful packages: textcomp,mathrsfs,amscd,epsfig,euscript,cancel

%%%%% Layout
% These numbers might depend on your printer. Check the margins and compare them to the Graduate Division's
% guidelines. If there's something off, try playing with the numbers...
%
% For chapters:
%     Must have a minimum of 1.5in margin on left and 1in on all other sides.  Where there are page numbers
%     (whether on top or bottom), must have one additional inch between the page number and the text, for a
%     total of 2in between the edge of the paper and the text.
% For frontmatter pages:
%     The same margin numbers generally work, except for the Title Page, so you will notice that we use
%     some numbers for \textheight and \footskip right here, and then change them below, right after
%     generating the Title Page.

\hoffset=.5in 
\oddsidemargin=0in   % = 1in because LaTeX adds 1in
\evensidemargin=0in  % = 1in because LaTeX adds 1in
\topmargin=0in       % = 1in because LaTeX adds 1in
\headheight=0in
\headsep=1in         % Distance from top of pagenum (for page numbers at top-right corner of page) to text
\footskip=1.2in      % Distance from bottom of text to the page number (for page number at bottom of page)
\textwidth=5.9in     % Should be 6in, but use 5.9in to be conservative
\textheight=8.0in    % Best for Title Page (will change after the Title Page)

\pagestyle{plain}

\doublespacing

%%%%% Style of theorems, definitions, examples, equations, etc.

\theoremstyle{plain} % Heading is bold, text italic.
\newtheorem{theorem}{Theorem}[chapter]
\newtheorem{lemma}[theorem]{Lemma}
\newtheorem{proposition}[theorem]{Proposition}
\newtheorem{corollary}[theorem]{Corollary}
\newtheorem{conjecture}{Conjecture}[chapter]

\theoremstyle{definition}  % Heading is bold, text is roman
\newtheorem{definition}{Definition}[chapter]
\newtheorem{example}{Example}[chapter]

\theoremstyle{remark}  % Heading is italic, text is roman
\newtheorem*{remark}{Remark}
\newtheorem*{note}{Note}
\newtheorem{claim}{Claim}[chapter]

%%%%% Appendix style

\renewcommand\appendix[1]{
\chapter*{#1}
\addcontentsline{toc}{chapter}{#1}
}

%%%%% Title page

\begin{document}

\pagenumbering{roman}
\thispagestyle{empty}

\[ \]
\vspace{-1.9in}

\begin{center}
{\mytitle}

\vspace{1.4in}

\singlespace{A thesis presented to the faculty of\\
San Francisco State University\\
In partial fulfillment of\\
The Requirements for\\ The Degree}

\vspace{.5in}

\singlespace{\mydegree \\ In\\ \myfield}

\vspace{3.1in}

{by \\[12pt] 
\myname \\[12pt]
San Francisco, California\\[12pt]
\thismonth
\thisyear}
\end{center}

\newpage
\setcounter{page}{1}
\textheight=7.1in    % For all pages after the Title Page -- try making this number smaller (7.0 or 6.9) if the bottom margins are too small
\footskip=1.1in      % Distance from bottom of text to the page number (for page number at bottom of page)
\thispagestyle{empty}

$\mbox{}$
\vspace{3in}
\begin{center}
\singlespace{
Copyright by\\ 
\myname \\
\thisyear
}
\end{center}

\newpage
\thispagestyle{empty}
\[ \]
\vspace{-1.8in}
\begin{center}
{CERTIFICATION OF APPROVAL}
\end{center}
\vspace{.6in}
\begin{quote}
I certify that I have read {\it \mytitle} by \myname and that in my opinion this work meets the criteria for approving a thesis submitted in partial fulfillment of the requirements
for the degree: \mydegree in \myfield at San Francisco State University.
\end{quote}

\vspace{0.75in}

\hspace*{\fill}\parbox{3.5in}{
\singlespace{

\hrule{\hspace{3.5in}} \\ 
Hui Yang, Ph.D.\\
Associate Professor of \myfield

\vspace{1in}

\hrule{\hspace{3.5in}} \\
Anagha Kulkarni, Ph.D.\\
Assistant Professor of \myfield 

\vspace{1in}

\hrule{\hspace{3.5in}} \\
Celia Graterol, M.P.H\\
Director of Information Systems\\
Metro College Success Program\\

}
}

\newpage
\thispagestyle{empty}
\[ \]
\vspace{-1.8in}
\begin{center}
{\mytitle} \\

\vspace{.3in}

\singlespace{
\myname \\
San Francisco State University \\ 
\thisyear \\
}

\end{center}

\vspace{.2in}

\onehalfspacing{\noindent
This work analyzes the curriculum-level factors that affected the persistence and graduation outcomes of over 2,000 students in the Metro College Success Program at San Francisco State University, and their control group counterparts.  This work addressed four questions: (1) how did the timing of students' Mathematics courses affect their performance, persistence, and graduation outcomes; (2) whether students who progressed farther through the prescribed foundation course sequences of the program exhibited higher persistence and graduation rates; (3) what were the most frequently taken sequences of courses, and whether students who progressed farther through those sequences exhibited higher persistence and graduation rates; and (4) whether greater progress was more important than other demographic and academic factors for predicting persistence and graduation.   We found that students who took their Math course in the second year showed higher fifth-term and seventh-term persistence than students who took it in the first year.  Also, students who progressed farther through course sequences consistently exhibited higher persistence and graduation rates.  Furthermore, a student's persistence was a more reliable predictor of graduation than other features.  Overall, these findings can potentially inform an institution's strategies for maximizing persistence and graduation by emphasizing a student's progress through the curriculum.
}

\vspace*{\fill}

\hspace*{\fill}

\noindent
I certify that the Abstract is a correct representation of the content of this thesis.

\vspace{.6in} 

\hrule{\hspace{3.75in}} \\[-10pt]
Chair, Thesis Committee 
\hspace{2.5in}
Date

\newpage
\[ \]
\vspace{-1.8in}
\begin{center}{ACKNOWLEDGMENTS}\end{center}

\vspace{.3in}
\begin{quote}
\noindent
I thank my advisor, Dr. Hui Yang, for generously sharing her time to provide valuable feedback and advice throughout this research.  Likewise, I wish to thank Ms. Celia Graterol at the Metro College Success Program for being a great colleague and friend during my time at Metro.  I also thank Dr. Anagha Kulkarni for being a member of my thesis committee and for her advice and guidance on helping introduce me to the graduate program at San Francisco State University, years ago.\\ \\
I would also like to thank all of the wonderful people at the Metro College Success Program for warmly welcoming me into the Metro team.  My time at Metro was not only a great work experience but also an opportunity to make new friendships.\\ \\
Last but not least, I would like to thank my mother Josephine, my sister Claudia, and Laura Carstensen for their ongoing support.
\end{quote}

\renewcommand{\contentsname}{\vspace{-1.8in} \begin{center} \normalsize \rm TABLE OF CONTENTS \end{center}}
\renewcommand{\listfigurename}{\vspace{-1.8in} \begin{center} \normalsize \rm LIST OF FIGURES \end{center}}
\renewcommand{\listtablename}{\vspace{-1.8in} \begin{center} \normalsize \rm LIST OF TABLES \end{center}}
\renewcommand{\cftchapfont}{\normalfont}
\renewcommand{\cftchappagefont}{\normalfont}
\renewcommand{\cftchapleader}{\cftdotfill{\cftdotsep}} % formatting commands for table of contents
\renewcommand{\cftsecfont}{\normalfont}
\renewcommand{\cftsecpagefont}{\normalfont}
\renewcommand{\cftsecleader}{\cftdotfill{\cftdotsep}}

\newpage \tableofcontents 
\newpage Table \hfill Page \listoftables % comment out if you don't use tables
\newpage Figure \hfill Page \listoffigures % comment out if you don't use figures

\newpage
\pagestyle{myheadings}
\pagenumbering{arabic} 
\setcounter{page}{1}
